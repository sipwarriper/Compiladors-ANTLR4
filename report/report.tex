\documentclass[11pt,a4paper,twoside]{article}
  \usepackage{a4wide}
  \usepackage{epsfig}
  \usepackage{amsmath}
  \usepackage{tabu}
  \usepackage{amsfonts}
  \usepackage{latexsym}
  \usepackage[utf8]{inputenc}
  \usepackage{listings}
  \usepackage{color}
  \usepackage{titlesec}    
  \usepackage{enumitem}
  \usepackage[catalan]{babel}
  \usepackage{newunicodechar}
  \usepackage{graphicx}
  \usepackage{subcaption}
  \usepackage{float}
  \usepackage{xcolor}
  \usepackage{makecell,multirow}

\setcounter{tocdepth}{4}
\setcounter{secnumdepth}{4}
  
\newunicodechar{Ŀ}{\L.}
\newunicodechar{ŀ}{\l.}


% \titleformat{\chapter}
%   {\normalfont\LARGE\bfseries}{\thechapter}{1em}{}
% \titlespacing*{\chapter}{0pt}{3.5ex plus 1ex minus .2ex}{2.3ex plus .2ex}

\definecolor{dkgreen}{rgb}{0,0.6,0}
\definecolor{gray}{rgb}{0.5,0.5,0.5}
\definecolor{mauve}{rgb}{0.58,0,0.82}


\usepackage{hyperref}
\hypersetup{
  colorlinks=false, %set true if you want colored links
  linktoc=all,     %set to all if you want both sections and subsections linked
  linkcolor=blue,  %choose some color if you want links to stand out
}



\setlength{\footskip}{50pt}
\setlength{\parindent}{0cm} \setlength{\oddsidemargin}{-0.5cm} \setlength{\evensidemargin}{-0.5cm}
\setlength{\textwidth}{17cm} \setlength{\textheight}{23cm} \setlength{\topmargin}{-1.5cm} \addtolength{\parskip}{2ex}
\setlength{\headsep}{1.5cm}

\title{Pràctica Compiladors: Compilador LANS}
\author{Ismael El Habri
	\and 
	Lluís Trilla
	}

\date{\today} 



\begin{document}
\maketitle

\section{Modificacions Part Léxica i Sintàctica}

\subsection{TK\_STRING\_LITERAL}

La nostra regla TK\_STRING\_LITERAL que ens serveix per identificar els literals de string és
\begin{lstlisting}
  TK_STRING_LITERAL: TK_OP_QUOTE (~('"'|'\n'|'\\') | ('\\"'))* TK_OP_QUOTE;
\end{lstlisting}
I anteriorment, TK\_OP\_QUOTE no éra un fragment, de manera que el lexer ens acceptava coses del estil "abc; . Ho vam corregir per tal que fós un fragment, i es va corregir l'error.

\subsection{TK\_CONST\_INT}
A la regla TK\_CONST\_INT anteriorment reconeixíem $0000$ però no $0001$, és a dir, només acceptavem 0s a l'esquerra en cas que el valor fós 0. Vam corregir la regla de manera que va quedar així:
\begin{lstlisting}
  TK_CONST_INT: (('0' | DIGIT)+);
\end{lstlisting}
I es va resoldre l'error esmentat. Això s'ha fet també amb els reals, per temes de consistència.
\begin{lstlisting}
  TK_CONST_REAL: ((('0' | DIGIT)+ '.') ('0' | DIGIT)+) 
    ('E' '-'? (DIGIT | '0') ('0' | DIGIT)*)?;
\end{lstlisting}

\subsection{TK\_MULTILINE\_COMMENTS}
A la regla TK\_MULTILINE\_COMMENTS continuem usant l'operador "*?". Si féssim servir
\begin{lstlisting}
(.*)?
\end{lstlisting}
  enlloc de
\begin{lstlisting}
(.)*?
\end{lstlisting}
    aleshores consideraríem cadenes com "/*comentari*/codiVàlid()/*comentari2*/"$ $
    com a un sòl comentari, el qual inclouria també codiValid().

\section{Estructura Taula de Símbols}
La taula de Símbols consisteix en un Mapa de Registres.

Els registres consisteixen en una classe amb els següents camps:

\begin{center}
  \begin{tabular}{ |c|c|c|}
    \hline
    \textbf{Camp} & \textbf{Tipus} & \textbf{Significat} \\
   \hline
    lexema & String & nom de l'entrada. \\
    \hline
    tipus & String & tipus del lexema. \\
    \hline
    tipusLexema & String & tipus de lexema que conté l'entrada. \\
    \hline
    \multirowcell{2}{campsAddicionals} & \multirowcell{2}{Vector de Pairs de dos Strings} & \multirowcell{2}{Camps addicionals, com ara els camps de les tuples \\o els paràmetres de les funcions i accions.} \\ &&\\
    \hline
    adreca & Long & Adreça on es situa l'entrada en la màquina. \\
   \hline
  \end{tabular}
  \end{center}

 Aquí una taula amb els possibles continguts dels camps:

 \begin{center}
  \begin{tabular}{ |c|c|c|c|}
    \hline
    Camp & Constant & Variable & Tupla \\
   \hline
   tipus & tipus & tipus & null \\
   tipusLexema & "const" & "var" & "tupla"  \\
   campsAddicionals & null & null & tipus i camps \\
   adreca & Adreça a la ConstantPool & Adreça variable & 0\\
   \hline
  \end{tabular}
  \end{center}

  \begin{center}
    \begin{tabular}{ |c|c|c|c|}
      \hline
      Camp & Alias & Acció/Funció & Vector\\
     \hline
     tipus  & tipus real & tipus retorn, void si és acció & tipus \\
     tipusLexema & "alias" & "funcio" & "vector" \\
     campsAddicionals & null & tipus paràmetres i entr o entrsor & null \\
     adreca & 0 & adreça funció/acció & adreça vector \\
     \hline
    \end{tabular}
    \end{center}

  A part, amb les tuples, s'han afegit registres com a variables per cada camp de la tupla, usant com a nom de lexema el següent format: "nomTupla.nomCamp". Com que sabem que no hi poden haver variables amb punts al mig, sabem que no hi haura conflictes.

\section{Parts no Obligatòries}

Hem fet el semàntic de les següents parts no obligatories:
\begin{itemize}
  \item Alies
  \item Tupla
  \item Vector
\end{itemize}

Pel que fa la generació de codi:

\begin{itemize}
  \item Alies
  \item Tupla
\end{itemize}

\section{Decisions}

La única decisió que hem fet ha estat que es puguin ficar zeros a l'esquerra en els enters i els reals.

\end{document}
